\documentclass[a4paper,11pt]{article}
\usepackage[a4paper, margin=8em]{geometry}

% usa i pacchetti per la scrittura in italiano
\usepackage[french,italian]{babel}
\usepackage[T1]{fontenc}
\usepackage[utf8]{inputenc}
\frenchspacing 

% usa i pacchetti per la formattazione matematica
\usepackage{amsmath, amssymb, amsthm, amsfonts}

% usa altri pacchetti
\usepackage{gensymb}
\usepackage{hyperref}
\usepackage{standalone}

\usepackage{colortbl}

\usepackage{xstring}
\usepackage{karnaugh-map}

% imposta il titolo
\title{Appunti Programmazione Avanzata}
\author{Luca Seggiani}
\date{2025}

% imposta lo stile
% usa helvetica
\usepackage[scaled]{helvet}
% usa palatino
\usepackage{palatino}
% usa un font monospazio guardabile
\usepackage{lmodern}

\renewcommand{\rmdefault}{ppl}
\renewcommand{\sfdefault}{phv}
\renewcommand{\ttdefault}{lmtt}

% circuiti
\usepackage{circuitikz}
\usetikzlibrary{babel}

% testo cerchiato
\newcommand*\circled[1]{\tikz[baseline=(char.base)]{
            \node[shape=circle,draw,inner sep=2pt] (char) {#1};}}

% disponi il titolo
\makeatletter
\renewcommand{\maketitle} {
	\begin{center} 
		\begin{minipage}[t]{.8\textwidth}
			\textsf{\huge\bfseries \@title} 
		\end{minipage}%
		\begin{minipage}[t]{.2\textwidth}
			\raggedleft \vspace{-1.65em}
			\textsf{\small \@author} \vfill
			\textsf{\small \@date}
		\end{minipage}
		\par
	\end{center}

	\thispagestyle{empty}
	\pagestyle{fancy}
}
\makeatother

% disponi teoremi
\usepackage{tcolorbox}
\newtcolorbox[auto counter, number within=section]{theorem}[2][]{%
	colback=blue!10, 
	colframe=blue!40!black, 
	sharp corners=northwest,
	fonttitle=\sffamily\bfseries, 
	title=Teorema~\thetcbcounter: #2, 
	#1
}

% disponi definizioni
\newtcolorbox[auto counter, number within=section]{definition}[2][]{%
	colback=red!10,
	colframe=red!40!black,
	sharp corners=northwest,
	fonttitle=\sffamily\bfseries,
	title=Definizione~\thetcbcounter: #2,
	#1
}

% disponi codice
\usepackage{listings}
\usepackage[table]{xcolor}

\definecolor{codegreen}{rgb}{0,0.6,0}
\definecolor{codegray}{rgb}{0.5,0.5,0.5}
\definecolor{codepurple}{rgb}{0.58,0,0.82}
\definecolor{backcolour}{rgb}{0.95,0.95,0.92}

\lstdefinestyle{codestyle}{
		backgroundcolor=\color{black!5}, 
		commentstyle=\color{codegreen},
		keywordstyle=\bfseries\color{magenta},
		numberstyle=\sffamily\tiny\color{black!60},
		stringstyle=\color{green!50!black},
		basicstyle=\ttfamily\footnotesize,
		breakatwhitespace=false,         
		breaklines=true,                 
		captionpos=b,                    
		keepspaces=true,                 
		numbers=left,                    
		numbersep=5pt,                  
		showspaces=false,                
		showstringspaces=false,
		showtabs=false,                  
		tabsize=2
}

\lstdefinestyle{shellstyle}{
		backgroundcolor=\color{black!5}, 
		basicstyle=\ttfamily\footnotesize\color{black}, 
		commentstyle=\color{black}, 
		keywordstyle=\color{black},
		numberstyle=\color{black!5},
		stringstyle=\color{black}, 
		showspaces=false,
		showstringspaces=false, 
		showtabs=false, 
		tabsize=2, 
		numbers=none, 
		breaklines=true
}


\lstdefinelanguage{assembler}{ 
  keywords={AAA, AAD, AAM, AAS, ADC, ADCB, ADCW, ADCL, ADD, ADDB, ADDW, ADDL, AND, ANDB, ANDW, ANDL,
        ARPL, BOUND, BSF, BSFL, BSFW, BSR, BSRL, BSRW, BSWAP, BT, BTC, BTCB, BTCW, BTCL, BTR, 
        BTRB, BTRW, BTRL, BTS, BTSB, BTSW, BTSL, CALL, CBW, CDQ, CLC, CLD, CLI, CLTS, CMC, CMP,
        CMPB, CMPW, CMPL, CMPS, CMPSB, CMPSD, CMPSW, CMPXCHG, CMPXCHGB, CMPXCHGW, CMPXCHGL,
        CMPXCHG8B, CPUID, CWDE, DAA, DAS, DEC, DECB, DECW, DECL, DIV, DIVB, DIVW, DIVL, ENTER,
        HLT, IDIV, IDIVB, IDIVW, IDIVL, IMUL, IMULB, IMULW, IMULL, IN, INB, INW, INL, INC, INCB,
        INCW, INCL, INS, INSB, INSD, INSW, INT, INT3, INTO, INVD, INVLPG, IRET, IRETD, JA, JAE,
        JB, JBE, JC, JCXZ, JE, JECXZ, JG, JGE, JL, JLE, JMP, JNA, JNAE, JNB, JNBE, JNC, JNE, JNG,
        JNGE, JNL, JNLE, JNO, JNP, JNS, JNZ, JO, JP, JPE, JPO, JS, JZ, LAHF, LAR, LCALL, LDS,
        LEA, LEAVE, LES, LFS, LGDT, LGS, LIDT, LMSW, LOCK, LODSB, LODSD, LODSW, LOOP, LOOPE,
        LOOPNE, LSL, LSS, LTR, MOV, MOVB, MOVW, MOVL, MOVSB, MOVSD, MOVSW, MOVSX, MOVSXB,
        MOVSXW, MOVSXL, MOVZX, MOVZXB, MOVZXW, MOVZXL, MUL, MULB, MULW, MULL, NEG, NEGB, NEGW,
        NEGL, NOP, NOT, NOTB, NOTW, NOTL, OR, ORB, ORW, ORL, OUT, OUTB, OUTW, OUTL, OUTSB, OUTSD,
        OUTSW, POP, POPL, POPW, POPB, POPA, POPAD, POPF, POPFD, PUSH, PUSHL, PUSHW, PUSHB, PUSHA, 
				PUSHAD, PUSHF, PUSHFD, RCL, RCLB, RCLW, MOVSL, MOVSB, MOVSW, STOSL, STOSB, STOSW, LODSB, LODSW,
				LODSL, INSB, INSW, INSL, OUTSB, OUTSL, OUTSW
        RCLL, RCR, RCRB, RCRW, RCRL, RDMSR, RDPMC, RDTSC, REP, REPE, REPNE, RET, ROL, ROLB, ROLW,
        ROLL, ROR, RORB, RORW, RORL, SAHF, SAL, SALB, SALW, SALL, SAR, SARB, SARW, SARL, SBB,
        SBBB, SBBW, SBBL, SCASB, SCASD, SCASW, SETA, SETAE, SETB, SETBE, SETC, SETE, SETG, SETGE,
        SETL, SETLE, SETNA, SETNAE, SETNB, SETNBE, SETNC, SETNE, SETNG, SETNGE, SETNL, SETNLE,
        SETNO, SETNP, SETNS, SETNZ, SETO, SETP, SETPE, SETPO, SETS, SETZ, SGDT, SHL, SHLB, SHLW,
        SHLL, SHLD, SHR, SHRB, SHRW, SHRL, SHRD, SIDT, SLDT, SMSW, STC, STD, STI, STOSB, STOSD,
        STOSW, STR, SUB, SUBB, SUBW, SUBL, TEST, TESTB, TESTW, TESTL, VERR, VERW, WAIT, WBINVD,
        XADD, XADDB, XADDW, XADDL, XCHG, XCHGB, XCHGW, XCHGL, XLAT, XLATB, XOR, XORB, XORW, XORL},
  keywordstyle=\color{blue}\bfseries,
  ndkeywordstyle=\color{darkgray}\bfseries,
  identifierstyle=\color{black},
  sensitive=false,
  comment=[l]{\#},
  morecomment=[s]{/*}{*/},
  commentstyle=\color{purple}\ttfamily,
  stringstyle=\color{red}\ttfamily,
  morestring=[b]',
  morestring=[b]"
}

\lstset{language=java, style=codestyle}

% disponi sezioni
\usepackage{titlesec}

\titleformat{\section}
	{\sffamily\Large\bfseries} 
	{\thesection}{1em}{} 
\titleformat{\subsection}
	{\sffamily\large\bfseries}   
	{\thesubsection}{1em}{} 
\titleformat{\subsubsection}
	{\sffamily\normalsize\bfseries} 
	{\thesubsubsection}{1em}{}

% tikz
\usepackage{tikz}

% float
\usepackage{float}

% grafici
\usepackage{pgfplots}
\pgfplotsset{width=10cm,compat=1.9}

% disponi alberi
\usepackage{forest}

\forestset{
	rectstyle/.style={
		for tree={rectangle,draw,font=\large\sffamily}
	},
	roundstyle/.style={
		for tree={circle,draw,font=\large}
	}
}

% disponi algoritmi
\usepackage{algorithm}
\usepackage{algorithmic}
\makeatletter
\renewcommand{\ALG@name}{Algoritmo}
\makeatother

% disponi numeri di pagina
\usepackage{fancyhdr}
\fancyhf{} 
\fancyfoot[L]{\sffamily{\thepage}}

\makeatletter
\fancyhead[L]{\raisebox{1ex}[0pt][0pt]{\sffamily{\@title \ \@date}}} 
\fancyhead[R]{\raisebox{1ex}[0pt][0pt]{\sffamily{\@author}}}
\makeatother

\begin{document}
% sezione (data)
\section{Lezione del 30-10-25}

% stili pagina
\thispagestyle{empty}
\pagestyle{fancy}

% testo
\subsection{Interrompere un thread}
Nelle prime versioni di Java esisteva un metodo sui thread, lo \lstinline|stop()|, che permetteva al chiamante di arrestare immediatamente l'esecuzione di un thread.
Questo meccanismo è stato deprecato quasi subito, in quanto pone seri rischi (ad esempio ci lascia facilmente lasciare strutture dati in stato inconsistente).

Quello che si consiglia di fare è quindi semplicemente prevedere variabili \lstinline|boolean running| nei thread, aggiornate da fuori ai thread, che la \lstinline|run()| dei thread controlla periodicamente per arrestarsi in maniera autonoma e pulita.

\subsubsection{Interruzioni thread}
Si possono mandare \texttbf{interrupt} sui thread, usando il meotodo \lstinline|interrupt()|.
Quando si invia un interrupt, può accadere una di 2 cose:
\begin{enumerate}
	\item Se il thread è bloccato con una \lstinline|sleep()|, una \lstinline|wait()| o una \lstinline|join()|, il thread esce e entra nei rispettivi blocchi \lstinline|catch| con l'\textit{InterruptedException}. In questo caso l'\textit{interrupt status} non viene controllato;
	\item Se il thread è in normale esecuzione, l'\textit{interrupt status} viene controllato.
\end{enumerate}

L'\textbf{interrupt status} è una condizione che può essere valutata dal thread con 2 metodi:
\begin{itemize}
	\item \lstinline|public static boolean interrupted()|: restituisce l'interrupt status e lo resetta;
	\item \lstinline|public boolean isInterrupted()|: restituisce l'interrupt status senza resettare.
\end{itemize}

\subsection{Classi wrapper}
Abbiamo introdotto in 1.3 le classi \textbf{wrapper} per i tipi primitivi.
Queste sono, in particolare:
\begin{table}[H]
	\center \rowcolors{2}{white}{black!10}
	\begin{tabular} { c | c }
		\bfseries Wrapper & \bfseries Tipi \\
		\hline
		Boolean & \lstinline|boolean|\\
		Byte & \lstinline|byte|\\
		Short & \lstinline|short|\\
		Int & \lstinline|int|\\
		Long & \lstinline|long|\\
		Float & \lstinline|float|\\
		Double & \lstinline|double|\\
		Character & \lstinline|char|\\
	\end{tabular}
\end{table}

Le classi wrapper possono essere create via \lstinline|new|, o molto più spesso attraverso il metodo \lstinline|valueOf()| (che accetta anche stringhe).
Sono \textbf{immutabili}, e perciò non possono cambiare (operazioni danno nuovi oggetti).

Solitamente rappresentano anche un modo per raggruppare costanti semanticamente coerenti col tipo rappresentato (per gli interi, \textit{intero minimo}, \textit{intero massimo}, ecc...).

\subsubsection{Autoboxing / unboxing}
I meccanismi di \textbf{autoboxing} e \textit{(auto)}\textbf{unboxing} permettono di convertire automaticamente da tipi primitivi a classi wrapper, senza aver bisogno di usare i metodi \lstinline|valueOf()|.

In particolare:
\begin{itemize}
	\item Il \textit{boxing} avviene quando una variabile o un argomento formale di tipo classe wrapper viene inizializzato con una corrispondente variabile di tipo primitivo: \lstinline|Integer i = 10|;
	\item L'\textit{unboxing} avviene, viceversa, quando una variabile o un argomento formale di tipo primitivo viene inizializzato con una corrispondente variabile di tipo classe wrapper: \lstinline|int i = Integer.valueOf(10)|.
\end{itemize}

\subsection{JMM}
Il \textbf{JMM} (\textit{Java Memory Model}) è il modello secondo cui Java mantiene le variabili (i campi) delle classi. 
Sostanzialmente, è l'insieme di regole che stabiliscono l'ordinamento degli accessi alla memoria e quando le modifiche sono visibili in modo garantito.
Ci è di interesse quando si parla di variabili condivise fra classi.

In Java, infatti, ogni valore condiviso fra thread deve essere acceduto in mutua esclusione (con \lstinline|synchronized|) per evitare interferenze.

L'acquisizione di un lock ha un costo, e in alcuni casi potrebbe non essere necesario: il linguaggio garantisce che le operazioni di lettura /scrittura su variabili (eccetto double e long) sono atomiche.
Quindi, se c'è un thread che modifica una variabile e altri che la leggono non ci sono problemi di interferenza.  
Questo però non assicura che un thread legga il valore più recente di una variabile (valore scritto da un altro thread), in assenza di meccanismi di sincronizzazione. Diversi fattori influiscono su quando una variabile modificata da un thread appare come tale ad altri thread. 

Il linguaggio prevede che certe "azioni" siano caratterizzate da una relazione "\textbf{happens-before}" ("\textit{avviene prima}"): 
\begin{itemize}
	\item Il rilascio di un lock su un oggetto "avviene prima" di ogni successiva acquisizione del lock sullo stesso oggetto;
	\item La scrittura su una variabile volatile "avviene prima" di ogni successiva lettura della stessa variabile;
	\item La chiamata a \lstinline|start()| "avviene prima" delle azioni nel thread che è stato fatto partire;
	\item Le operazioni in un thread "avvengono prima" della join() su tale thread. 
\end{itemize}
\end{document}
